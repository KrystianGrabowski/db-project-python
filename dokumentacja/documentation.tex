\documentclass[a4paper]{article}
\usepackage[utf8]{inputenc}
\usepackage[a4paper, left=2.5cm, right=2.5cm, top=2cm, bottom=1cm]{geometry}
\usepackage{polski}
\usepackage[polish]{babel}
\usepackage{indentfirst}



\title{Projekt: System Zarządzania Partią Polityczną}
\author{Krystian Grabowski} 
\begin{document}
\maketitle
\section{Opis}
Program został napisany w języku Python v3.6.7. Składa się z plików
\begin{itemize}
    \itemsep0em
    \item[-] \textbf{main.py} \\ zawiera wszystkie funkcje interpretujące dane i zarządzające bazą
    \item[-] \textbf{db\_init.sql} \\ zawiera polecenia tworzące niezbędne elementy bazy, triggery oraz funkcje
\end{itemize}
  
\section{Wymagane pakiety}
Do poprawnego uruchomienia wymagane jest kilka dodatkowych pakietów, są to m.in:
\begin{itemize}
    \itemsep0em
    \item psycopg2
    \item json
    \item argparse
    \item sys
\end{itemize}


\section{Uruchomienie}
\subsection{Argumenty wywołania}
\begin{itemize}
    \itemsep0em
    \item[] \textbf{-h, --help} \\ pomoc
    \item[] \textbf{-i, --init} \\  inicjalizacja bazy danych
    \item[] \textbf{-g, --debug} \\  aktywuje tryb debugownia, 
        informacje o błędach będą wyświetlane
    \item[] \textbf{-f FILENAME, --filename FILENAME} \\ wczytuje plik FILENAME 
    oraz interpretuje obiekty JSON w nim zawarte
\end{itemize}
Jeżeli nie podano flagi -f program po uruchomieniu zacznie wczytywać dane ze standardowego wejścia.
\subsection{Pierwsze uruchomienie}
Podczas pierwszego uruchomienia należy dodać flagę --init. Dzięki niej zostanią 
utworzone wszystkie niezbędne elementy w bazie danych.

\begin{itemize}
    \itemsep0em
    \item[-] \textbf{python3 main.py --init} \\ 
        Rozpoczyna inicjalizację bazy danych czytając ze strumienia wejściowego.
    \item[-] \textbf{python3 main.py --init -f FILENAME} \\  
        Rozpoczyna inicjalizację bazy czytając z pliku o nazwie FILENAME
    \item[-] \textbf{python3 main.py --init $<$ FILENAME} \\  
        Przekazuje na standardowe wejście dane z pliku FILENAME oraz inicjalizuje bazę danych
     
\end{itemize}

\subsection{Kolejne uruchomienia}
\begin{itemize}
    \itemsep0em
    \item[-] \textbf{python3 main.py} \\ 
        Rozpoczyna odczyt ze strumienia wejściowego.
    \item[-] \textbf{python3 main.py -f FILENAME} \\  
        Rozpoczyna odczyt z pliku o nazwie FILENAME
    \item [-] \textbf{python3 main.py $<$ FILENAME} \\
        Przekazuje na standardowe wejście dane z pliku FILENAME

\end{itemize}

\end{document}